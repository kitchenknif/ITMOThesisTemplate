% !TEX program = xelatex
% !TEX root = thesis.tex
\documentclass{itmo-thesis}

\usepackage{color}
\usepackage{gensymb}

\newcommand{\hl}[1]{
{\color{yellow} #1}}

\setmainlanguage{russian}
\begin{document}

\filltitle{ru}{
    chair              = {Кафедра нанофотоники и метаматериалов},
    title              = {Исследование различных методик прокрастинации при необходимости оформлять тексты в ТеХе},
    type               = {master},
    position           = {студента},
    group              = {V4140},
    coursenum          = {12.04.03},
    course             = {Нанофотоника и Метаматериалы},
    author             = {Дмитриев Павел Алексеевич},
    supervisorPosition = {к.\,ф.-м.\,н.},
    supervisor         = {Самый С.О.К.},
    reviewerPosition   = {Тоже},
    reviewer           = {Фрукт},
    chairHeadPosition  = {д.\,ф.-м.\,н.},
    chairHead          = {Маракуйя},
}
\maketitle
\tableofcontents

\section{Section McSectionFace}
\section{Раздел Разделочный}
    \subsection{И это ещё не конец}
        \subsubsection{Всё ещё не нащупали дно}
            \paragraph{Вот дно, но снизу могут постучать}




\newpage

%\setlength{\parskip}{-3pt}
%\renewcommand\bibsection{\nsection{Литература}}
%\let\oldbibliography\thebibliography
%\renewcommand{\thebibliography}[1]{%
%  \oldbibliography{#1}
%  \setlength{\itemsep}{6pt}
%}
%\def\BibPrefix{}

\nocite{*}
%\bibliographystyle{disser_m}
\bibliographystyle{ugost2008ls}
\bibliography{references}
\end{document}
